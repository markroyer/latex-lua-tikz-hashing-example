\documentclass{beamer}
\usetheme{default}

\usepackage{calc}
\usepackage{forloop}

\usepackage{tikz}

\usetikzlibrary{arrows,fit,calc,shapes.multipart,shadows,chains,snakes}

\usepackage{lstcustom}

\lstset{
  language=C++,
  style=eclipse,
  showspaces=false, 
  numbers=none,
  frame=,
  captionpos=b,
  breaklines=true,
  % From http://tex.stackexchange.com/questions/116534/lstlisting-line-wrapping
  postbreak=\raisebox{0ex}[0ex][0ex]{\ensuremath{\color{red}\hookrightarrow\space}},
  escapechar=\$
}



\title{Hashing}
\subtitle{Linear Probing Example}
\date{}
\author{Mark Royer}


%%%%%%%%%%%%%%%%%%%%%%%%%%%%%%% Start Lua stuff %%%%%%%%%%%%%%%%%%%%%%%%%%%%%%%%%

\directlua{dofile("array.lua")}
\newcommand*{\displayArray}[3]{%
  \directlua{tex.sprint(displayArray(\unexpanded{{#1}},#2,"\luaescapestring{#3}"))}
}

%%%%%%%%%%%%%%%%%%%%%%%%%%%%%%% End Lua stuff %%%%%%%%%%%%%%%%%%%%%%%%%%%%%%%%%%%

\begin{document}

% Uncomment this for lua debug info and replace \directlua with
% \directluadebug

\newwrite\luadebug
\immediate\openout\luadebug luadebug.lua
\AtEndDocument{\immediate\closeout\luadebug}
\newcommand\directluadebug{\immediate\write\luadebug}

% End Lua debug info

%%%%%%%%%%%%%%%%%%%%%%%%%%%%%%%%%%%%%%%%%%%%%%%%%%%%%%%%%%%%%%%%%%%%%%%%%%%%%%%%%

\begin{frame}
  \titlepage
\end{frame}

%%%%%%%%%%%%%%%%%%%%%%%%%%%%%%%%%%%%%%%%%%%%%%%%%%%%%%%%%%%%%%%%%%%%%%%%%%%%%%%%%

\begin{frame}[fragile]{Hash tables}

\textbf{Hash Tables} are containers that support constant time ($O(1)$)
performance for the following operations:

\begin{itemize}

  \item Insert
  \item Remove
  \item Find

\end{itemize}

\end{frame}

%%%%%%%%%%%%%%%%%%%%%%%%%%%%%%%%%%%%%%%%%%%%%%%%%%%%%%%%%%%%%%%%%%%%%%%%%%%%%%%%%

\begin{frame}[fragile]{Searching with hash tables}

\begin{itemize}
\item Suppose you are modeling customers, and that they are uniquely
  identified by a telephone number.  The telephone number is the
  \textbf{key} used for searching.

\item If an array could be established with indices $0 \dots 9999999$,
  the key could be used as the array index yielding direct access in
  constant run-time.  

\item For example, if the customers array is called
  \lstinline$customers$ then

  \begin{itemize}

  \item \lstinline$customers[5813522]$ is the customer record with key 5813522
    
  \item  \lstinline$customers[5814433]$ is the customer record with key 5814433
    
  \end{itemize}
  
\end{itemize}

\end{frame}


%%%%%%%%%%%%%%%%%%%%%%%%%%%%%%%%%%%%%%%%%%%%%%%%%%%%%%%%%%%%%%%%%%%%%%%%%%%%%%%%%

\begin{frame}[fragile]{Decreasing the array size}

\begin{itemize}
\item Since many applications will not contain all 10,000,000 telephone numbers,
  an array is used with length approximately the number of customers
  in the data set, say 1000.  

\item This saves $9999000\text{int}\times\frac{4\text{B}}{1\text{int}}=39996000\text{B}\approx38\text{MB}$!

\item This type of structure  is called a \textbf{hash table}.

\item The phone number cannot now serve as the array index directly
  because most of the numbers are greater than 999.

\item Instead a \textbf{hash function} is used to map the phone number
  to an index in the range $0 \dots 999$.

\end{itemize}

\end{frame}

%%%%%%%%%%%%%%%%%%%%%%%%%%%%%%%%%%%%%%%%%%%%%%%%%%%%%%%%%%%%%%%%%%%%%%%%%%%%%%%%%

\begin{frame}[fragile]{Simple hash function}

\begin{itemize}
\item A simple hash function might use the first 3 digits of the phone number.

\begin{lstlisting}[numbers=none]
int hash(int key) {	
  return   key / 10000;	
}
\end{lstlisting}

\item The \textbf{hash function} defined above maps
  telephone numbers into the array for storage and look-up.

\item \lstinline$hash(4740997) == 474$ \\ \lstinline$customers[474]$ is
  the location of the customer with telephone number 4740997.

\item \lstinline$hash(5130795) == 513$ \\ \lstinline$customers[513]$ is
  the location of the customer with telephone number 5130795.

\end{itemize}

\end{frame}

%%%%%%%%%%%%%%%%%%%%%%%%%%%%%%%%%%%%%%%%%%%%%%%%%%%%%%%%%%%%%%%%%%%%%%%%%%%%%%%%%

\begin{frame}[fragile]{Collisions}

\begin{itemize}

\item Unfortunately, this simple algorithm easily leads to \textbf{collisions}.

\item Collisions occur when multiple keys map to the same array index
  (\textbf{synonyms}). For example, the following telephone numbers
  all hash to the same location.

\begin{center}
\lstinline$hash(5810997) == 581$ \\ 
\lstinline$hash(5810795) == 581$ \\ 
\lstinline$hash(5811145) == 581$ \\
\end{center}


\item Although collisions are inevitable, the better hash functions
  minimize their occurrence.

\end{itemize}

\end{frame}

%%%%%%%%%%%%%%%%%%%%%%%%%%%%%%%%%%%%%%%%%%%%%%%%%%%%%%%%%%%%%%%%%%%%%%%%%%%%%%%%%

\begin{frame}[fragile]{Division method}

\begin{itemize}

\item There are many hash function methods.  A common technique is the
  \textbf{division method}.

\item This function causes fewer collisions if the length of the table
  is a prime number (eg. 1009).

\item The revised hashing function

\begin{lstlisting}[numbers=none]
const int SIZE = 1009;

int hash(int key) {
  return (key % SIZE);
}
\end{lstlisting}

\begin{center}
\lstinline$hash(5810997) == 166$ \\ 
\lstinline$hash(5810795) == 973$ \\ 
\lstinline$hash(5811145) == 314$ \\
\end{center}

\end{itemize}

\end{frame}

%%%%%%%%%%%%%%%%%%%%%%%%%%%%%%%%%%%%%%%%%%%%%%%%%%%%%%%%%%%%%%%%%%%%%%%%%%%%%%%%%

\begin{frame}[fragile]{Collision resolution}

\begin{itemize}

\item When a collision occurs, two issues must be addressed:

\begin{enumerate}

\item Where will the item be inserted?

\item How will the search algorithm find this item since it is not in its expected position?

\end{enumerate}

\end{itemize}

\end{frame}

%%%%%%%%%%%%%%%%%%%%%%%%%%%%%%%%%%%%%%%%%%%%%%%%%%%%%%%%%%%%%%%%%%%%%%%%%%%%%%%%%

\begin{frame}[fragile]{Open address collision resolution}

\begin{itemize}

\item Use some function to find an alternative index in the case of a
  collision.  A collision occurs on key $k$ when
  \lstinline$table[hash(k)]$ is already used.

\item The \textbf{linear probing} strategy tests array elements sequentially
  until it finds an empty slot (here filled with flag -1).

\item Each attempt to insert into a location in the table is called a \textbf{probe}.

\end{itemize}

\begin{center}
\resizebox{10.0cm}{!}{%
  \begin{tikzpicture}
   \displayArray{-1,-1,-1,-1,-1,-1,-1,-1,-1,-1,-1}{-1}{table:}
  \end{tikzpicture}
}%
\end{center}


\end{frame}

%%%%%%%%%%%%%%%%%%%%%%%%%%%%%%%%%%%%%%%%%%%%%%%%%%%%%%%%%%%%%%%%%%%%%%%%%%%%%%%%%

\begin{frame}[fragile]{Linear probing example}

\begin{itemize}

\item Insert elements \textcolor{yellow}{45}, 8, 35, 64, 3, 14 into a \textbf{hash table} of size 11.

\item \textbf{Insert} 45

\item \lstinline$hash(45)$\ $\Rightarrow 1$

\item 1 is empty, insert 45.

\end{itemize}

\begin{center}
\resizebox{10.0cm}{!}{%
  \begin{tikzpicture}
   \displayArray{-1,45,-1,-1,-1,-1,-1,-1,-1,-1,-1}{1}{table:}
  \end{tikzpicture}
}%
\end{center}


\end{frame}

%%%%%%%%%%%%%%%%%%%%%%%%%%%%%%%%%%%%%%%%%%%%%%%%%%%%%%%%%%%%%%%%%%%%%%%%%%%%%%%%%

\begin{frame}[fragile]{Linear probing example}

\begin{itemize}

\item Insert elements 45, \textcolor{yellow}{8}, 35, 64, 3, 14 into a
  \textbf{hash table} of size 11.

\item \textbf{Insert} 8

\item \lstinline$hash(8)$\ $\Rightarrow 8$

\item 8 is empty, insert 8.

\end{itemize}

\begin{center}
\resizebox{10.0cm}{!}{%
  \begin{tikzpicture}
   \displayArray{-1,45,-1,-1,-1,-1,-1,-1,8,-1,-1}{8}{table:}
  \end{tikzpicture}
}%
\end{center}


\end{frame}

%%%%%%%%%%%%%%%%%%%%%%%%%%%%%%%%%%%%%%%%%%%%%%%%%%%%%%%%%%%%%%%%%%%%%%%%%%%%%%%%%

\begin{frame}[fragile]{Linear probing example}

\begin{itemize}

\item Insert elements 45, 8, \textcolor{yellow}{35}, 64, 3, 14 into a
  \textbf{hash table} of size 11.

\item \textbf{Insert} 35

\item \lstinline$hash(35)$\ $\Rightarrow 2$

\item 2 is empty, insert 35.

\end{itemize}

\begin{center}
\resizebox{10.0cm}{!}{%
  \begin{tikzpicture}
   \displayArray{-1,45,35,-1,-1,-1,-1,-1,8,-1,-1}{2}{table:}
  \end{tikzpicture}
}%
\end{center}


\end{frame}

%%%%%%%%%%%%%%%%%%%%%%%%%%%%%%%%%%%%%%%%%%%%%%%%%%%%%%%%%%%%%%%%%%%%%%%%%%%%%%%%%

\begin{frame}[fragile]{Linear probing example}

\begin{itemize}

\item Insert elements 45, 8, 35, \textcolor{yellow}{64}, 12, 14 into a \textbf{hash table} of size 11.

\item \textbf{Insert} 64

\item \lstinline$hash(64)$\ $\Rightarrow 9$

\item 9 is empty, insert 64.

\end{itemize}

\begin{center}
\resizebox{10.0cm}{!}{%
  \begin{tikzpicture}
   \displayArray{-1,45,35,-1,-1,-1,-1,-1,8,64,-1}{9}{table:}
  \end{tikzpicture}
}%
\end{center}


\end{frame}

%%%%%%%%%%%%%%%%%%%%%%%%%%%%%%%%%%%%%%%%%%%%%%%%%%%%%%%%%%%%%%%%%%%%%%%%%%%%%%%%%

\begin{frame}[fragile]{Linear probing example}

\begin{itemize}

\item Insert elements 45, 8, 35, 64, \textcolor{yellow}{12}, 14 into a \textbf{hash table} of size 11.

\item \textbf{Insert} 12

\item \lstinline$hash(12)$\ $\Rightarrow 1$

\item 1 is \textbf{not empty}, 
\only<2->{ attempt hash(12)+1, 2 is \textbf{not empty},}
\only<3->{ attempt hash(12)+2, 3 is empty, insert 12.}

\end{itemize}

\begin{center}
\resizebox{10.0cm}{!}{%
  \begin{tikzpicture}
   \only<1>{\displayArray{-1,45,35,-1,-1,-1,-1,-1,8,64,-1}{1}{table:}}
   \only<2>{\displayArray{-1,45,35,-1,-1,-1,-1,-1,8,64,-1}{2}{table:}}
   \only<3>{\displayArray{-1,45,35,12,-1,-1,-1,-1,8,64,-1}{3}{table:}}
  \end{tikzpicture}
}%
\end{center}


\end{frame}

%%%%%%%%%%%%%%%%%%%%%%%%%%%%%%%%%%%%%%%%%%%%%%%%%%%%%%%%%%%%%%%%%%%%%%%%%%%%%%%%%

\begin{frame}[fragile]{Linear probing example}

\begin{itemize}

\item Insert elements 45, 8, 35, 64, 12, \textcolor{yellow}{14} into a
  \textbf{hash table} of size 11.

\item \textbf{Insert} 14

\item \lstinline$hash(14)$\ $\Rightarrow 3$

\item 3 is \textbf{not empty},
\only<2->{ attempt hash(14)+1, 4 is empty, insert 14.}

\end{itemize}

\begin{center}
\resizebox{10.0cm}{!}{%
  \begin{tikzpicture}
\only<1>{\displayArray{-1,45,35,12,-1,-1,-1,-1,8,64,-1}{3}{table:}}
\only<2>{\displayArray{-1,45,35,12,14,-1,-1,-1,8,64,-1}{4}{table:}}
  \end{tikzpicture}
}%
\end{center}


\end{frame}

%%%%%%%%%%%%%%%%%%%%%%%%%%%%%%%%%%%%%%%%%%%%%%%%%%%%%%%%%%%%%%%%%%%%%%%%%%%%%%%%%


\end{document}
